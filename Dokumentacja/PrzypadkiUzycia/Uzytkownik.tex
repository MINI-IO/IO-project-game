\documentclass[../Dokumentacja.tex]{subfiles}
\begin{document}
\subsection{Użytkownik}
Użytkownik jest jest osobą używającą The Project Game.  Inicjuje wszystkie moduły występujące w projekcie. Posiada wymienione poniżej możliwości ingerencji w działanie poszczególnych modułów.
\subsubsection{Moduł serwera}
\begin{itemize}
    \item Włączanie serwera
    \begin{itemize}
    	\item Ustawianie portów do połączeń z GM oraz agentami
    	\item Ustawianie limitu ilości połączeń
    \end{itemize}
    \item Wyłączanie serwera
\end{itemize}

\subsubsection{Moduł agenta}
\begin{itemize}
	\item Inicjowanie agenta
	\begin{itemize}
		\item Tworzenie i łączenie agenta z serwerem posługując się portami ustawionymi w module serwer
		\item Wprowadzanie parametrów połączenia z serwerem: rodzaj połączenia, numer portu, zgodnych portami ustawionymi w module serwer
		\item Wprowadzanie do której drużyny należy agent (Red, Blue)
		\item Wybranie strategii używanej podczas gry przez agenta spośród zaimplementowanych w module agenta. Od wybranej strategii zależy sposób podejmowania przez agenta takich decyzji jak: przesunięcie się na planszy, opuszczenie fragmentu, czy odpowiedź innemu agentowi na zapytanie
	\end{itemize}
	\item Wyłączanie agenta
	\begin{itemize}
		\item Wysłanie informacji o wyłączeniu agenta do serwera
	\end{itemize}
\end{itemize}

\subsubsection{Moduł GM}
\begin{itemize}
	\item Inicjowanie GM
	\begin{itemize}
		\item Wprowadzanie parametrów połączeń: rodzaj połączenia, numer portu, zgodnych portami ustawionymi w module serwer
	\end{itemize}
	\item Konfigurowanie parametrów rozgrywki
	\begin{itemize}
		\item Ilość celów w polu bramkowym
		\item Wymiary pola bramkowego
		\item Opóźnienie w wykonywaniu ruchów przez agenta
		\item Wymiary planszy
		\item Maksymalna ilość agentów
		\item Prawdopodobieństwo, że pojawiający się fragment jest fragmentem fikcyjnym
		\item Częstotliwość generowania nowego fragmentu na planszy
	\end{itemize}
	\item Rozpoczynanie rozgrywki
	\begin{itemize}
		\item Wysłanie informacji o rozpoczęciu do serwera
		\item Utworzenie planszy i rozgrywki na podstawie wcześniej wprowadzonych parametrów, lub w przypadku niewprowadzenia parametrów, na podstawie parametrów domyślnych
	\end{itemize}
	\item Wyświetlanie planszy
	\begin{itemize}
		\item Pobieranie stanu planszy
		\item Pobieranie statystyk rozgrywki
	\end{itemize}
	\item Zakończenie rozgrywki
	\begin{itemize}
		\item Wysłanie informacji o zakończeniu do serwera
		\item Wyświetlanie statystyk rozgrywki
	\end{itemize}
\end{itemize}
\end{document}